\documentclass[journal,12pt,twocolumn]{IEEEtran}
%
\usepackage{setspace}
\usepackage{gensymb}
\usepackage{tkz-euclide} 
\usepackage{textcomp}
\usepackage{standalone}
\usetikzlibrary{calc}
\usepackage{float}
%\doublespacing
\singlespacing

%\usepackage{graphicx}
%\usepackage{amssymb}
%\usepackage{relsize}
\usepackage[cmex10]{amsmath}
%\usepackage{amsthm}
%\interdisplaylinepenalty=2500
%\savesymbol{iint}
%\usepackage{txfonts}
%\restoresymbol{TXF}{iint}
%\usepackage{wasysym}
\usepackage{amsthm}
%\usepackage{iithtlc}
\usepackage{mathrsfs}
\usepackage{txfonts}
\usepackage{stfloats}
\usepackage{bm}
\usepackage{cite}
\usepackage{cases}
\usepackage{subfig}
%\usepackage{xtab}
\usepackage{longtable}
\usepackage{multirow}
%\usepackage{algorithm}
%\usepackage{algpseudocode}
\usepackage{enumitem}
\usepackage{mathtools}
\usepackage{steinmetz}
\usepackage{tikz}
\usepackage{circuitikz}
\usepackage{verbatim}
\usepackage{tfrupee}
\usepackage[breaklinks=true]{hyperref}
%\usepackage{stmaryrd}
\usepackage{tkz-euclide} % loads  TikZ and tkz-base
%\usetkzobj{all}
\usetikzlibrary{calc,math}
\usepackage{listings}
    \usepackage{color}                                            %%
    \usepackage{array}                                            %%
    \usepackage{longtable}                                        %%
    \usepackage{calc}                                             %%
    \usepackage{multirow}                                         %%
    \usepackage{hhline}                                           %%
    \usepackage{ifthen}                                           %%
  %optionally (for landscape tables embedded in another document): %%
    \usepackage{lscape}     
\usepackage{multicol}
\usepackage{chngcntr}
\usepackage{romannum}
%\usepackage{enumerate}
%\usepackage{wasysym}
%\newcounter{MYtempeqncnt}
\DeclareMathOperator*{\Res}{Res}
%\renewcommand{\baselinestretch}{2}
\renewcommand\thesection{\arabic{section}}
\renewcommand\thesubsection{\thesection.\arabic{subsection}}
\renewcommand\thesubsubsection{\thesubsection.\arabic{subsubsection}}

\renewcommand\thesectiondis{\arabic{section}}
\renewcommand\thesubsectiondis{\thesectiondis.\arabic{subsection}}
\renewcommand\thesubsubsectiondis{\thesubsectiondis.\arabic{subsubsection}}

% correct bad hyphenation here
\hyphenation{op-tical net-works semi-conduc-tor}
\def\inputGnumericTable{}                                 %%

\lstset{
%language=C,
frame=single, 
breaklines=true,
columns=fullflexible
}
%\lstset{
%language=tex,
%frame=single, 
%breaklines=true
%}

\begin{document}
%


\newtheorem{theorem}{Theorem}[section]
\newtheorem{problem}{Problem}
\newtheorem{proposition}{Proposition}[section]
\newtheorem{lemma}{Lemma}[section]
\newtheorem{corollary}[theorem]{Corollary}
\newtheorem{example}{Example}[section]
\newtheorem{definition}[problem]{Definition}
%\newtheorem{thm}{Theorem}[section] 
%\newtheorem{defn}[thm]{Definition}
%\newtheorem{algorithm}{Algorithm}[section]
%\newtheorem{cor}{Corollary}
\newcommand{\BEQA}{\begin{eqnarray}}
\newcommand{\EEQA}{\end{eqnarray}}
\newcommand{\define}{\stackrel{\triangle}{=}}
\bibliographystyle{IEEEtran}
%\bibliographystyle{ieeetr}
\providecommand{\mbf}{\mathbf}
\providecommand{\pr}[1]{\ensuremath{\Pr\left(#1\right)}}
\providecommand{\qfunc}[1]{\ensuremath{Q\left(#1\right)}}
\providecommand{\sbrak}[1]{\ensuremath{{}\left[#1\right]}}
\providecommand{\lsbrak}[1]{\ensuremath{{}\left[#1\right.}}
\providecommand{\rsbrak}[1]{\ensuremath{{}\left.#1\right]}}
\providecommand{\brak}[1]{\ensuremath{\left(#1\right)}}
\providecommand{\lbrak}[1]{\ensuremath{\left(#1\right.}}
\providecommand{\rbrak}[1]{\ensuremath{\left.#1\right)}}
\providecommand{\cbrak}[1]{\ensuremath{\left\{#1\right\}}}
\providecommand{\lcbrak}[1]{\ensuremath{\left\{#1\right.}}
\providecommand{\rcbrak}[1]{\ensuremath{\left.#1\right\}}}
\theoremstyle{remark}
\newtheorem{rem}{Remark}
\newcommand{\sgn}{\mathop{\mathrm{sgn}}}
\providecommand{\abs}[1]{\left\vert#1\right\vert}
\providecommand{\res}[1]{\Res\displaylimits_{#1}} 
\providecommand{\norm}[1]{\left\lVert#1\right\rVert}
%\providecommand{\norm}[1]{\lVert#1\rVert}
\providecommand{\mtx}[1]{\mathbf{#1}}
\providecommand{\mean}[1]{E\left[ #1 \right]}
\providecommand{\fourier}{\overset{\mathcal{F}}{ \rightleftharpoons}}
%\providecommand{\hilbert}{\overset{\mathcal{H}}{ \rightleftharpoons}}
\providecommand{\system}{\overset{\mathcal{H}}{ \longleftrightarrow}}
	%\newcommand{\solution}[2]{\textbf{Solution:}{#1}}
\newcommand{\solution}{\noindent \textbf{Solution: }}
\newcommand{\cosec}{\,\text{cosec}\,}
\providecommand{\dec}[2]{\ensuremath{\overset{#1}{\underset{#2}{\gtrless}}}}
\newcommand{\myvec}[1]{\ensuremath{\begin{pmatrix}#1\end{pmatrix}}}
\newcommand{\mydet}[1]{\ensuremath{\begin{vmatrix}#1\end{vmatrix}}}
%\numberwithin{equation}{section}
\numberwithin{equation}{subsection}
%\numberwithin{problem}{section}
%\numberwithin{definition}{section}
\makeatletter
\@addtoreset{figure}{problem}
\makeatother
\let\StandardTheFigure\thefigure
\let\vec\mathbf
%\renewcommand{\thefigure}{\theproblem.\arabic{figure}}
\renewcommand{\thefigure}{\theproblem}
%\setlist[enumerate,1]{before=\renewcommand\theequation{\theenumi.\arabic{equation}}
%\counterwithin{equation}{enumi}
%\renewcommand{\theequation}{\arabic{subsection}.\arabic{equation}}
\def\putbox#1#2#3{\makebox[0in][l]{\makebox[#1][l]{}\raisebox{\baselineskip}[0in][0in]{\raisebox{#2}[0in][0in]{#3}}}}
     \def\rightbox#1{\makebox[0in][r]{#1}}
     \def\centbox#1{\makebox[0in]{#1}}
     \def\topbox#1{\raisebox{-\baselineskip}[0in][0in]{#1}}
     \def\midbox#1{\raisebox{-0.5\baselineskip}[0in][0in]{#1}}
\vspace{3cm}
\title{Assignment 4}
\author{Vinay kumar}
\maketitle
\newpage
Download all python codes
and latex-tikz codes from 
%
\begin{lstlisting}
https://github.com/jvinaykumar12/EE5609/tree/master/Assignment4
\end{lstlisting}
%
\section{Problem}
$AD$ is the altitude of a isosceles $\triangle ABC$ in which $AB = AC$.Show that 
\begin{enumerate}
\item $AD$ bisects $BC$
\item $AD$ bisects $\angle A$
\end{enumerate}

\section{Explanation}

Given $AD$ is altitude of the $\triangle ABC$. Therefore

\begin{align}
(\vec{B}-\vec{C})^T(\vec{A}-\vec{D}) = 0
\label{eq:a0}
\end{align}


\begin{figure}[!ht] \label{fig:triangle_abc}
\centering
\resizebox{\columnwidth}{!}{\input{picture.tex}}
\caption{$\triangle ABC$ with $AD$ as altitude}
\end{figure} 

\begin{multline}
\norm{\vec{A-B}}^2  = (\vec{A}-\vec{B})^T(\vec{A}-\vec{B}) \\
= (\vec{A}-\vec{B})^T((\vec{A}-\vec{D})-(\vec{B}-\vec{D})) \\
= (\vec{A}-\vec{B})^T(\vec{A}-\vec{D})-(\vec{A}-\vec{B})^T(\vec{B}-\vec{D}) \\
= \vec{A}^T(\vec{A}-\vec{D})-\vec{B}^T(\vec{A}-\vec{D})-(\vec{A}-\vec{B})^T(\vec{B}-\vec{D}) 
\label{eq:a1}
\end{multline}
Similarly for AC,
\begin{multline}
\norm{\vec{A-C}}^2 = (\vec{A}-\vec{C})^T(\vec{A}-\vec{C}) \\
= (\vec{A}-\vec{C})^T((\vec{A}-\vec{D})-(\vec{C}-\vec{D})) \\
= (\vec{A}-\vec{C})^T(\vec{A}-\vec{D})-(\vec{A}-\vec{C})^T(\vec{C}-\vec{D}) \\
= \vec{A}^T(\vec{A}-\vec{D})-\vec{C}^T(\vec{A}-\vec{D})-(\vec{A}-\vec{C})^T(\vec{C}-\vec{D}) 
\label{eq:a2}
\end{multline}
Given AB = AC. Therefore
\begin{align}
\norm{\vec{A-B}} = \norm{\vec{A-C}}
\label{eq:a4}
\end{align}
By equating \eqref{eq:a1} and \eqref{eq:a2}
\begin{multline}
\vec{B}^T(\vec{A}-\vec{D})+(\vec{A}-\vec{B})^T(\vec{B}-\vec{D}) = \\ 
\vec{C}^T(\vec{A}-\vec{D})+(\vec{A}-\vec{C})^T(\vec{C}-\vec{D}) \\
\implies (\vec{B}-\vec{C})^T(\vec{A}-\vec{D})+(\vec{A}-\vec{B})^T(\vec{B}-\vec{D})  \\
= (\vec{A}-\vec{C})^T(\vec{C}-\vec{D}) \quad\text{From \eqref{eq:a0}} \\
(\vec{A}-\vec{B})^T(\vec{B}-\vec{D}) = (\vec{A}-\vec{C})^T(\vec{C}-\vec{D})
\label{eq:a5}
\end{multline}
Since $\triangle ABC$ is isosceles angle ABD is equal to angle ACD
\begin{align}
\frac{(\vec{A}-\vec{B})^T(\vec{B}-\vec{D})}{\norm{\vec{A}-\vec{B}}\norm{\vec{B}-\vec{D}}} =
\frac{(\vec{A}-\vec{C})^T(\vec{C}-\vec{D})}{\norm{\vec{A}-\vec{C}}\norm{\vec{C}-\vec{D}}} 
\end{align}
By refering the values from \ref{eq:a4} and \ref{eq:a5}
\begin{align}
\norm{\vec{B}-\vec{D}} = \norm{\vec{C}-\vec{D}}
\label{eq:a6}
\end{align}
Therefore BD = DC. In $\triangle ABD$
\begin{align}
\cos{\angle DAB} = \frac{(\vec{D}-\vec{A})^T(\vec{A}-\vec{B})}{\norm{\vec{D}-\vec{A}}\norm{\vec{A}-\vec{B}}} \\ =
\frac{(\vec{D}-\vec{A})^T(\vec{A}-\vec{C}+\vec{C}-\vec{B})}{\norm{\vec{D}-\vec{A}}\norm{\vec{A}-\vec{B}}} \\ =
\frac{(\vec{D}-\vec{A})^T(\vec{A}-\vec{C})+(\vec{D}-\vec{A})^T(\vec{C}-\vec{B})}{\norm{\vec{D}-\vec{A}}\norm{\vec{A}-\vec{B}}}
\end{align}
By refering the values from \eqref{eq:a0} \eqref{eq:a4}
\begin{align}
\frac{(\vec{D}-\vec{A})^T(\vec{A}-\vec{C})}{\norm{\vec{D}-\vec{A}}\norm{\vec{A}-\vec{C}}} = \cos{\angle DAC}
\end{align}
Therefore angle DAC is equal to angle DAB. Thus AD is the angular bisector of angle A.
\end{document}